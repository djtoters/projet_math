\documentclass{article}
sepackage[utf8]{inputenc}
sepackage{geometry}
\geometry{margin=1in}
	itle{rapport\_1}
egin{document}
\maketitle

\section*{Objectif général du projet}
L’objectif de ce projet est d’analyser l’évolution de l’épidémie de COVID-19 dans différentes communes à l’aide de modèles mathématiques avancés, en particulier :
egin{itemize}
  \item Les Chaînes de Markov à États Cachés (HMM, Hidden Markov Models)
  \item Les Modèles de Mélange Gaussien (GMM, Gaussian Mixture Models, via l’algorithme EM)
nd{itemize}
Ces modèles permettent de détecter automatiquement des “phases” ou “régimes” épidémiques, de comparer différentes communes, et même de faire de la prédiction.

\section*{Données utilisées}
egin{itemize}
  \item 	extbf{Source} : Fichier Filtered\_Covid19\_data.csv
  \item 	extbf{Colonnes principales} :
    egin{itemize}
      \item DATE : date de l’observation
      \item TX\_DESCR\_FR : nom de la commune
      \item CASES\_PER\_10K\_MA : nombre de cas COVID pour 10 000 habitants (moyenne mobile)
    nd{itemize}
nd{itemize}

\section*{Modélisation mathématique}
\subsection*{HMM (Hidden Markov Model)}
Principe : On suppose que l’évolution des cas COVID est gouvernée par des “états cachés” (ex : faible circulation, pic, décroissance…) qui ne sont pas observés directement.
Le modèle apprend :
egin{itemize}
  \item Les probabilités de transition entre états cachés.
  \item Les caractéristiques de chaque état (moyenne des cas, etc.).
nd{itemize}
Utilité : Détecter automatiquement les différentes phases de l’épidémie.

\subsection*{GMM (Gaussian Mixture Model, via EM)}
Principe : On suppose que les observations proviennent d’un mélange de plusieurs distributions gaussiennes (une par “phase”).
L’algorithme EM (Expectation-Maximization) permet d’estimer les paramètres de ces distributions et d’assigner chaque observation à une phase.

\section*{Critère de sélection du modèle : le BIC}
BIC (Bayesian Information Criterion) : Permet de choisir le nombre optimal d’états cachés (HMM) ou de composantes (GMM) en équilibrant qualité d’ajustement et complexité du modèle.
Plus le BIC est bas, mieux c’est.

\section*{Scripts et fonctionnalités développés}
egin{itemize}
  \item 	extbf{Analyse HMM sur une commune} (hmm\_Bxl.py) : Sélection automatique du nombre d’états cachés par BIC, visualisation des phases détectées, analyse mathématique détaillée.
  \item 	extbf{Analyse GMM/EM sur une commune} (gmm\_bxl.py) : Application d’un GMM, sélection du nombre de phases par BIC, explication du rôle de l’algorithme EM.
  \item 	extbf{Prédiction HMM sur une même commune} (hmm\_bxl\_forecast.py) : L’utilisateur choisit la durée d’entraînement et de prédiction, comparaison graphique et calcul de l’erreur (RMSE).
  \item 	extbf{Transférabilité HMM entre communes} (hmm\_commune\_transfer.py) : Sélection assistée des communes, entraînement sur une commune, application à une autre, comparaison graphique et RMSE, vérification automatique des données.
nd{itemize}

\section*{Ergonomie et robustesse}
egin{itemize}
  \item Sélection conviviale des communes (par numéro ou nom).
  \item Vérification automatique de la présence de données.
  \item Prompts interactifs pour choisir les paramètres d’analyse.
nd{itemize}

\section*{Interprétation des résultats}
egin{itemize}
  \item Visualisation : Les phases détectées sont colorées sur les courbes de cas.
  \item Analyse mathématique : Pour chaque phase, on obtient la moyenne des cas, la durée, et les dates des épisodes majeurs.
  \item Prédiction : On peut évaluer la capacité du modèle à anticiper l’évolution de l’épidémie, ou à transférer la connaissance d’une commune à une autre.
nd{itemize}

\section*{Ce que le projet permet de démontrer}
egin{itemize}
  \item La puissance des modèles probabilistes pour détecter des régimes cachés dans des séries temporelles réelles.
  \item L’importance de la sélection de modèle (BIC) pour éviter le surajustement.
  \item Les limites de la prédiction et de la transférabilité entre territoires (communes).
  \item L’intérêt d’une analyse mathématique rigoureuse et d’une visualisation claire pour l’interprétation des résultats.
nd{itemize}

\section*{Prolongements possibles}
egin{itemize}
  \item Tester d’autres modèles (changepoint detection, modèles non-gaussiens…)
  \item Ajouter des variables explicatives (météo, mesures sanitaires…)
  \item Automatiser l’analyse sur toutes les communes
  \item Créer un notebook interactif pour l’exploration pédagogique
nd{itemize}

\section*{Résumé}
Ce projet met en œuvre des outils mathématiques avancés pour comprendre, visualiser et prédire l’évolution de l’épidémie de COVID-19 à l’échelle locale, tout en restant accessible et interactif pour l’utilisateur.

nd{document}
